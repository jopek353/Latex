\documentclass[12pt]{article}
\usepackage[T1]{fontenc}
\usepackage{amsmath}
\usepackage{amssymb}
\usepackage{graphicx}
\graphicspath{ {./images/} }
\title{Czarna dziura}
\author{Piotr Jadczak}
\date{28 grudnia 2019 r.}

\begin{document}
\maketitle
\newpage
\section{Opis}

\textbf{Czarna dziura} \textendash{} obszar czasoprzestrzeni, kt\'{o}rego z uwagi na wp\l{}yw grawitacji, nic (\l{}\k{a}cznie ze \'{s}wiat\l{}em) nie mo\.{z}e opu\'{s}ci\'{c}\footnote[1]{Wald 1984 , s. 299–300.}. Zgodnie z og\'{o}ln\k{a} teori\k{a} wzgl\k{e}dno\'{s}ci, do jej powstania niezb\k{e}dne jest nagromadzenie dostatecznie du\.{z}ej masy w odpowiednio ma\l{}ej obj\k{e}to\'{s}ci. Czarn\k{a} dziur\k{e} otacza matematycznie zdefiniowana powierzchnia nazywana horyzontem zdarze\'{n}, kt\'{o}ra wyznacza granic\k{e} bez powrotu. Nazywa si\k{e} j\k{a} ,,czarn\k{a}\textquotedblright{}, poniewa\.{z} poch\l{}ania ca\l{}kowicie \'{s}wiat\l{}o trafiaj\k{a}ce w horyzont, nie odbijaj\k{a}c niczego, zupe\l{}nie jak cia\l{}o doskonale czarne w termodynamice\footnote[2]{P.C.W. Davies. Thermodynamics of Black Holes. ,,Reports on Progress in Physics\textquotedblright{}. 41 (8), s. 1313\textendash{}1355, 1978. DOI: 10.1088/0034-4885/41/8/004}. Mechanika kwantowa przewiduje, \.{z}e czarne dziury emituj\k{a} promieniowanie jak cia\l{}o doskonale czarne o niezerowej temperaturze. Temperatura ta jest odwrotnie proporcjonalna do masy czarnej dziury, co sprawia, \.{z}e bardzo trudno je zaobserwowa\'{c} w wypadku czarnych dziur o masie gwiazdowej b\k{a}d\'{z} wi\k{e}kszych\cite{theory}.

\begin{figure}[ht]
\begin{center}
\includegraphics[scale=0.5]{bh2.eps}
\caption{Symulowany widok czarnej dziury}
\label{rys_model_bh}
\end{center}
\end{figure}


\newpage
\section{Historia}
\subsection{Pocz\k{a}tek}
Ide\k{e}, \.{z}e mo\.{z}e istnie\'{c} tak masywne cia\l{}o, i\.{z} nawet \'{s}wiat\l{}o nie mo\.{z}e z niego uciec, postulowa\l{} angielski geolog John Michell w roku 1783 w pracy przes\l{}anej do Royal Society. W tym czasie istnia\l{}a teoria grawitacji Isaaca Newtona i poj\k{e}cie pr\k{e}dko\'{s}ci ucieczki. Michell rozwa\.{z}a\l{}, i\.{z} w kosmosie mo\.{z}e istnie\'{c} wiele tego typu obiekt\'{o}w\footnote[3]{Hawking 2003, s. 37.}.

W roku 1796 francuski matematyk Pierre Simon de Laplace propagowa\l{} t\k{e} sam\k{a} ide\k{e} w swojej ksi\k{a}\.{z}ce Exposition du Systeme du Monde (niestety znikn\k{e}\l{}a w p\'{o}\'{z}niejszych wydaniach\footnote[4]{Hawking 2003, s. 38.}).
\subsection{Wiek XX}
Ta idea nie cieszy\l{}a si\k{e} du\.{z}ym zainteresowaniem w XIX wieku, poniewa\.{z} \'{s}wiat\l{}o uwa\.{z}ano za bezmasow\k{a} fal\k{e} niepodlegaj\k{a}c\k{a} grawitacji. Nied\l{}ugo po opublikowaniu w roku 1905 szczeg\'{o}lnej teorii wzgl\k{e}dno\'{s}ci, Einstein zacz\k{a}\l{} rozwa\.{z}a\'{c} wp\l{}yw grawitacji na \'{s}wiat\l{}o.

Najpierw pokaza\l{}, \.{z}e grawitacja oddzia\l{}uje na propagacj\k{e} fal \newline elektromagnetycznych, a w roku 1915 sformu\l{}owa\l{} og\'{o}ln\k{a} teori\k{e} wzgl\k{e}dno\'{s}ci. Kilka miesi\k{e}cy p\'{o}\'{z}niej, Karl Schwarzschild znalaz\l{} rozwi\k{a}zanie r\'{o}wna\'{n} tej teorii opisuj\k{a}cych obiekt maj\k{a}cy posta\'{c} masy skupionej w jednym punkcie, kt\'{o}ry bardzo silnie odkszta\l{}ca czasoprzestrze\'{n}. W roku 1931 Chandrasekhar na przyk\l{}adzie bia\l{}ego kar\l{}a pokaza\l{}, \.{z}e powy\.{z}ej pewnej granicznej masy nic nie jest w stanie powstrzyma\'{c} kolapsu gwiazdy.
\subsection{Wsp\'{o}\l{}czesno\'{s}\'{c}}
W 1939 roku Robert Oppenheimer i Hartland Snyder pokazali, \.{z}e masywna gwiazda mo\.{z}e ulec kolapsowi grawitacyjnemu. Taki obiekt nazwano ,,zamro\.{z}on\k{a} gwiazd\k{a}\textquotedblright{}, poniewa\.{z} dla dalekiego obserwatora kolaps b\k{e}dzie zwalnia\l{}. Idea ta nie wywo\l{}a\l{}a du\.{z}ego zainteresowania a\.{z} do lat 60. Zainteresowanie ni\k{a} wzros\l{}o z chwil\k{a} odkrycia pulsar\'{o}w w 1967 roku. Tu\.{z} po tym w 1969 John Wheeler zaproponowa\l{} nazw\k{e} ,,czarna dziura\textquotedblright{}.


\newpage
\section{Opis matematyczny}
Poniewa\.{z} zakrzywienie czasoprzestrzeni jest odczuwane jako si\l{}a grawitacji, czasem m\'{o}wi si\k{e} potocznie, \.{z}e czarn\k{a} dziur\k{e} stanowi materia \'{s}ci\'{s}ni\k{e}ta tak, \.{z}e si\l{}a grawitacji, z jak\k{a} oddzia\l{}uje ona na sam\k{a} siebie, nie mo\.{z}e by\'{c} zr\'{o}wnowa\.{z}ona przez si\l{}y wewn\k{e}trzne (ci\'{s}nienie). Jest to uproszczenie o tyle, \.{z}e w my\'{s}l r\'{o}wna\'{n} Einsteina ci\'{s}nienie daje wk\l{}ad wsp\'{o}\l{}dzia\l{}aj\k{a}cy z si\l{}\k{a} grawitacji (czyli wzrost ci\'{s}nienia przyspiesza, a nie spowalnia powstanie czarnej dziury).
\paragraph{R\'{o}wnanie Einsteina}
Istnienie czarnych dziur wynika z r\'{o}wnania Einsteina Og\'{o}lnej Teorii Wzgl\k{e}dno\'{s}ci, cho\'{c} w historii fizyki ju\.{z} wcze\'{s}niej pojawi\l{}a si\k{e} hipotetyczna idea masy tak wielkiej, \.{z}e nawet \'{s}wiat\l{}o nie mog\l{}oby si\k{e} od niej oddali\'{c}. R\'{o}wnania Og\'{o}lnej Teorii Wzgl\k{e}dno\'{s}ci (OTW), z kt\'{o}rych wynika istnienie czarnych dziur, maj\k{a} posta\'{c}:

\begin{equation*}
R_{\mu \nu} - \frac{1}{2} g_{\mu \nu}\,R + g_{\mu \nu} \Lambda = 
 \frac{8 \pi G}{c^4} T_{\mu \nu}
\end{equation*}
gdzie:
\begin{description}
    \item $g$: tensor metryczny
    \item $R_{\mu \nu}$: tensor krzywizny Ricciego
    \item $R$: skalar krzywizny Ricciego
    \item $T_{\mu \nu}$: tensor energii-p\k{e}du
\end{description}


\paragraph{Rozwi\k{a}zanie Schwarzschilda}
Jednym z dos\l{}ownie kilku znanych rozwi\k{a}za\'{n} tych r\'{o}wna\'{n} jest rozwi\k{a}zanie Schwarzschilda \textendash{} metryka czasoprzestrzeni dana wzorem:

\begin{equation*}
ds^2 = e^{\nu(r)} dt^2 - e^{\lambda(r)} dr^2 - r^2 d\Omega^2 = \left(1 - \frac{2M}{r}\right) dt^2 - \left(1 - \frac{2M}{r}\right)^{-1}dr^2 - r^2 d\Omega^2
\end{equation*}
\newpage

\section{W\l{}a\'{s}ciwo\'{s}ci czarnej dziury}
Istniej\k{a} teorie, wed\l{}ug kt\'{o}rych przej\'{s}cie obiektu przez horyzont zdarze\'{n} zwi\k{a}zane jest z ca\l{}kowitym znikni\k{e}ciem zawartej w tym obiekcie informacji\cite{astrof}.

Z matematycznego punktu widzenia fakt ten sprowadza si\k{e} do stwierdzenia, \.{z}e do opisu czarnej dziury wystarczy poda\'{c} jej mas\k{e}, \l{}adunek oraz moment p\k{e}du\cite{relativ}. Dla poszczeg\'{o}lnych kombinacji tych trzech warto\'{s}ci sformu\l{}owano nast\k{e}puj\k{a}ce rozwi\k{a}zania r\'{o}wna\'{n} opisuj\k{a}cych czarn\k{a} dziur\k{e}:
 
\begin{enumerate}
	\item Schwarzschilda \textendash{} tylko masa niezerowa,
	\item Reissnera-Nordstr\"{o}ma \textendash{} \l{}adunek, masa niezerowa, brak 		momentu p\k{e}du,
	\item Kerra \textendash{} masa i moment p\k{e}du niezerowy, brak \l{}adunku,
	\item Kerra-Newmanna \textendash{} \l{}adunek, masa, moment p\k{e}du 				niezerowe.
\end{enumerate}

\subsection{Klasyfikacja czarnych dziur}

\begin{table}[h]
\textbf{\caption{Black hole classifications}}
\label{tab:blackholeclass}
\begin{center}
\renewcommand{\arraystretch}{2}
\begin{tabular}{|l|c|c|}
\hline
\textbf{Class} & \textbf{Approx. mass} & \textbf{Approx. radius} \\
\hline
Supermassive black hole & $10^5-10^{10} M_{sun}$ & $0.001-400$ AU \\
\hline
Intermediate-mass black hole & $10^3 M_{sun}$ & $10^3$ km $\approx R_{Earth}$  \\
\hline
Stellar black hole & $10 M_{sun}$ & $30$ km \\
\hline
Micro black hole & up to $M_{moon}$ & up to $0.1$ mm \\
\hline
\end{tabular}
\end{center}
\end{table}
\newpage
\section{Pierwszy obraz czarnej dziury}

\begin{figure}[ht]
\begin{center}
\includegraphics[scale=0.3]{bh1.eps}
\caption{Obraz obiektu M87}
\label{obraz_M87}
\end{center}
\end{figure}
\noindent
10 kwietnia 2019 naukowcy z programu EHT opublikowali pierwszy w historii obraz czarnej dziury, znajduj\k{a}cej si\k{e} w centrum galaktyki M87.
\newpage
\begin{center}
\section{Spis tre\'{s}ci}
\end{center}
\tableofcontents
\listoftables
\listoffigures

\newpage

\bibliography{bibl}
\bibliographystyle{plain}



\end{document}